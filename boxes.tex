
%%%% Box Definition %%%%
%\newtcolorbox{prob}[1]{
	%	% Set box style
	%	sidebyside align=top,
	%	% Dimensions and layout
	%	width=\textwidth,
	%	toptitle=2.5pt,
	%	bottomtitle=2.5pt,
	%	% Coloring
	%	colbacktitle=gray!30,
	%	coltitle=black,
	%	colback=white,
	%	colframe=black,
	%	% Title formatting
	%	title={ #1 },
	%	fonttitle=\large\bfseries
	%}

\usepackage[most, breakable, many]{tcolorbox} % nice looking boxes for example/question 
\newtcbtheorem{example}{Example}
{%
	colback = myexamplebg
	,breakable
	,colframe = myexamplefr
	,coltitle = myexampleti
	,boxrule = 1pt
	,sharp corners
	,detach title
	,before upper=\tcbtitle\par\smallskip
	,fonttitle = {\bfseries\fontfamily{lmss}\selectfont}
	,description font = \mdseries
	,separator sign none
}
{ex}
\makeatletter
\newtcbtheorem{question}{Question}
{
	enhanced,
	breakable,
	colback=blue!7,
	colframe=blue!90,
	attach boxed title to top left={yshift*=-\tcboxedtitleheight},
	fonttitle={\bfseries\fontfamily{lmss}\selectfont\hphantom{i}},
	boxed title size=title,
	boxed title style={%
		sharp corners,
		rounded corners=northwest,
		colback=tcbcolframe,
		boxrule=3pt,
	},
	underlay boxed title={%
		\path[fill=tcbcolframe] (title.south west)--(title.south east)
		to[out=0, in=180] ([xshift=5mm]title.east)--
		(title.center-|frame.east)
		[rounded corners=\kvtcb@arc] |-
		(frame.north) -| cycle;
	}
}{def}
\makeatother
\newcommand*{\Scale}[2][4]{\scalebox{#1}{\ensuremath{#2}}}   % Scales a mathmode object by a number